
Introduction to IS-IS
    History of IS-IS
        ISO 10589
        However, IS-IS is neutral regarding the type of network addresses for which it can route, and was easily extended to support IPv4 routing, using mechanisms described in RFC 1195, and later IPv6 as specified in RFC 5308
        
    Interior Gateway Protocol
    Link-state routing protocol
    Similarities with OSPF Protocol
        Both OSPF and IS-IS routers build a topological representation of the network. This map indicates the subnets which each IS-IS router can reach, and the lowest-cost (shortest) path to a subnet is used to forward traffic.
        IS-IS differs from OSPF in the way that "areas" are defined and routed between. IS-IS routers are designated as being: Level 1 (intra-area); Level 2 (inter area); or Level 1–2 (both). Routing information is exchanged between Level 1 routers and other Level 1 routers of the same area, and Level 2 routers can only form relationships and exchange information with other Level 2 routers. Level 1–2 routers exchange information with both levels and are used to connect the inter area routers with the intra area routers.

        In OSPF, areas are delineated on the interface such that an area border router (ABR) is actually in two or more areas at once, effectively creating the borders between areas inside the ABR, whereas in IS-IS area borders are in between routers, designated as Level 2 or Level 1–2. The result is that an IS-IS router is only ever a part of a single area.

        IS-IS also does not require Area 0 (Area Zero) to be the backbone area through which all inter-area traffic must pass. The logical view is that OSPF creates something of a spider web or star topology of many areas all attached directly to Area Zero and IS-IS, by contrast, creates a logical topology of a backbone of Level 2 routers with branches of Level 1–2 and Level 1 routers forming the individual areas.

        IS-IS also differs from OSPF in the methods by which it reliably floods topology and topology change information through the network.
        
    Djikstra Algorithm
        Both support Classless Inter-Domain Routing, can use multicast to discover neighboring routers using hello packets, and can support authentication of routing updates.
Integrated IS-IS Configuration
     The TCP/IP implementation, known as "Integrated IS-IS" or "Dual IS-IS", is described in RFC 1195.
IS-IS Authentication
IS-IS DIS and Pseudonode
IS-IS Metric on Cisco IOS
IS-IS Redistribution
IS-IS Summarization
IS-IS Filtering
IS-IS Route Leaking
